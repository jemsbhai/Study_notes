\documentclass[]{article}
\usepackage[T1]{fontenc}
\usepackage{lmodern}
\usepackage{amssymb,amsmath}
\usepackage{ifxetex,ifluatex}
\usepackage{fixltx2e} % provides \textsubscript
% use upquote if available, for straight quotes in verbatim environments
\IfFileExists{upquote.sty}{\usepackage{upquote}}{}
\ifnum 0\ifxetex 1\fi\ifluatex 1\fi=0 % if pdftex
  \usepackage[utf8]{inputenc}
\else % if luatex or xelatex
  \ifxetex
    \usepackage{mathspec}
    \usepackage{xltxtra,xunicode}
  \else
    \usepackage{fontspec}
  \fi
  \defaultfontfeatures{Mapping=tex-text,Scale=MatchLowercase}
  \newcommand{\euro}{€}
\fi
% use microtype if available
\IfFileExists{microtype.sty}{\usepackage{microtype}}{}
\ifxetex
  \usepackage[setpagesize=false, % page size defined by xetex
              unicode=false, % unicode breaks when used with xetex
              xetex]{hyperref}
\else
  \usepackage[unicode=true]{hyperref}
\fi
\hypersetup{breaklinks=true,
            bookmarks=true,
            pdfauthor={},
            pdftitle={},
            colorlinks=true,
            citecolor=blue,
            urlcolor=blue,
            linkcolor=magenta,
            pdfborder={0 0 0}}
\urlstyle{same}  % don't use monospace font for urls
\setlength{\parindent}{0pt}
\setlength{\parskip}{6pt plus 2pt minus 1pt}
\setlength{\emergencystretch}{3em}  % prevent overfull lines
\setcounter{secnumdepth}{0}

\author{}
\date{}

\begin{document}

\section{Longitudianl Data Analysis}\label{longitudianl-data-analysis}

\subsubsection{Cross-sectional v.s. Longitudianl
study}\label{cross-sectional-v.s.-longitudianl-study}

\textbf{Cross-sectional (CS)} study can be reduced from the longitudinal
study (LS) if the number of measures per subject is equal to one, ie
$n_i=1$. \[Y_{i1}=\beta_cx_{i1}+\varepsilon_{i1}, i=1,\cdots, m\]

\begin{itemize}
\itemsep1pt\parskip0pt\parsep0pt
\item
  $\beta_c$ represents the diff in average $Y$ across two
  sub-populations which differ by one unit in $x$.
\end{itemize}

With repeated measures, above model becomes LDA model:
\[Y_{ij}=\beta_cx_{i1}+\beta_L(x_{ij}-x_{i1})+\varepsilon_{ij}, i=1,\cdots,m; j=1,\cdots,n_i\]

\begin{itemize}
\item
  Notice that $\beta_L$ represents the expected change in $Y$ over time
  per one unit change in $x$, w.r.t its baseline value:
  \[(Y_{ij}-Y_{i1})=\beta_L(x_{ij}-x_{i1})+\varepsilon_{ij}-\varepsilon_{i1}\]
\item
  When $n=1$, above two models are identical.
\item
  It's more common that $\beta_C$ and $\beta_L$ have the same sign.
  However, it may exist that they have opposite sign.
\item
  In CS the basis is a comparison of individuals with a particular value
  of $x$ to others with a different value
\item
  In LDA each person is his/her own control. $\beta_L$ is estimated by
  comparing a person's response at two times assuming that $x$ changes
  over time.
\item
  Estimation of $\beta_C$ is confouned by unmeasured individual
  characteristic; while estimation of $\beta_L$ is less likely to be
  affected by unmeasured confounding. (since, if the confounders are not
  time-varing, they will be cancelled out when doing model (2)-(1)).
\end{itemize}

\end{document}
